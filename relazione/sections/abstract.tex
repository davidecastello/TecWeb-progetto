\documentclass[../relazione.tex]{subfiles}

\begin{document}
\section{Abstract}
		Il gruppo ha deciso di realizzare un sito web per il ristorante Sakura, un ristorante giapponese di Bassano del Grappa.\\
		Il tutto è nato dalla necessità di sviluppare un sito web per il progetto del corso di Tecnologie Web e, in contemporanea, l'incontro con Ming, il padrone del suddetto ristorante.\\
		Dopo avergli parlato delle nozioni di accessibilità e di posizionamento sui motori di ricerca apprese a lezione, Ming è rimasto davvero impressionato e ha chiesto al gruppo di sfruttare l'occasione del progetto didattico per realizzare il nuovo sito del ristorante.\\
		Durante la fase iniziale, abbiamo deciso di dedicare una parte di tempo per analizzare il sito attuale del ristorante, che è reperibile al seguente link: \\\\\url{http://www.ristorantesakura.com}
		\\\\Abbiamo riscontrato i seguenti problemi nel sito attuale:
		\begin{itemize}
			\item non vi è una completa separazione tra contenuto e presentazione;
			\item non vi è una completa separazione tra contenuto e comportamento;
			\item il layout è tabellare;
			\item il codice \texttt{XHTML} è dichiarato non vado dal validatore di W3C;
			\item tutto il menù del sito è riportato sotto forma di più immagini una di seguito all'altra, rendendolo completamente inaccessibile ad uno screen reader;
			\item non viene indicata attraverso l'attributo \texttt{xml:lang} la presenza di parole straniere, che sono moltissime all'interno del sito.
		\end{itemize}
		La domanda che ci siamo posti durante questa fase iniziale di analisi è: quali vantaggi può portare il nostro sito web al ristorante?\\
		Sicuramente permetterà di raggiungere un maggior numeri di utenti: tenendo conto di questo, riteniamo fondamentale che il sito in questione sia accessibile per permettere a chiunque di visitarlo.\\
		I social network sono molto utili in questo caso, ma non permettono un'interfaccia personalizzata, impedendo di mostrare la propria personalità attraverso il design.\\
		Un'altra considerazione che abbiamo fatto è stata che se l'esperienza dell'utente sul sito è piacevole e i contenuti sono ben ordinati e facilmente reperibil, l'utente è invogliato ad usare il sito per le operazioni più frequenti: ad esempio, guardare il menù e i prezzi prima di ordinare, o guardare dal sito il numero di telefono e far partire da lì la chiamata.
\end{document}