\documentclass[../relazione.tex]{subfiles}

\begin{document}
\section{Analisi dell'utenza}
	\subsection{A}
	\begin{itemize}
		\item quando si parla di utenza l'attenzione si sposta sull'interfaccia: come il sito appare all'utente ma anche come l'utente interagisce con esso
		\item essendo un ristorante si potrebbe supporre che una qualsiasi persona sia interessata ad informarsi sul ristorante, ordinare take-away, ma non è proprio così
		\item chiaramente non è possibile spiegare ogni termine giapponese ad un utente che non conosce nulla della cucina nipponica, quindi si presuppone che l'utente abbia già avuto un'esperienza passata con la cucina giapponese
		\item una grossa fetta di utenti è composta dai ragazzi più giovani della zona attorno a Bassano del Grappa (dove è situato il ristorante) che sono assolutamente innamorati della cucina giapponese e tendono ad andare in compagnia o ordinare take-away: per questo motivo il gruppo ha deciso di definire lo stile per il mobile, tenendo conto che la maggior parte degli accessi al sito da parte di ragazzi giovani verrà effettuata da dispositivo mobile, smartphone o tablet che esso sia
		\item questi dati sono stati ricavati dall'esperienza personale ma anche da un confronto con lo stesso proprietario del Ristorante Sakura: il signor Ming ci ha dato molte informazioni interessanti riguardo il mondo della cucina giapponese che abbiamo deciso di sfruttare a nostro vantaggio per ottenere un sito giusto per il suo bacino d'utenza
		\item per cercare di ricordare lo stile nipponico abbiamo definito un background legno (...)
		\item per lo stesso motivo, è stata definita una pagina di accesso negato (nel caso in cui si provi ad accedere dalla barra degli indirizzi all'area amministratore del sito o nel caso in cui scada la sessione) che mostra due simpatici sushi
	\end{itemize}

	\subsection{B}
\end{document}