\documentclass[../relazione.tex]{subfiles}

\begin{document}
\section{Suddivisione dei ruoli}
	\begin{itemize}
		\item inizialmente è stata effettuata un'analisi dettagliata del sito attuale del ristorante e della possibile utenza del sito, dei bisogni degli utente, dei bisogni del proprietario del ristorante
		\item una volta ottenuti questi dati è iniziata una fase di progettazione del sito web dove tutti i membri hanno lavorato assieme per cercare di definire più in dettaglio possibile le varie pagine necessarie, le funzionalità da fornire, ecc
	\end{itemize}
	Successivamento, il gruppo si è diviso gli incarichi.\\
	Andrea Tombolato si è occupato dei seguenti compiti:
	\begin{itemize}
		\item Configurazione dell’ambiente di lavoro
		\item Codifica e manutenzione del codice CSS
		\item Codifica e manutenzione del codice XHTML
		\item Esecuzione dei test sulla validazione del codice XHTML
		\item Esecuzione dei test sulla validazione del codice CSS
	\end{itemize}
	Davide Castello si è occupato dei seguenti compiti:
	\begin{itemize}
		\item Definizione della struttura dei file XML del database
		\item Definizione degli schemi associati ai file XML
		\item Esecuzione dei test sulla validazione del codice XML associato allo schema
		\item Esecuzione dei test sull’accessibilità della parte pubblica e dell'area amministratore
		\item Esecuzione dei test sulla compatibilità con più browser
		\item Stesura della relazione, raccogliendo le scelte progettuali importanti dagli altri membri del gruppo e tenendo conto della fase di analisi svolta ad inizio progetto
	\end{itemize}
	Eduard Bicego si è occupato dei seguenti compiti:
	\begin{itemize}
		\item Codifica e manutenzione degli script Perl/CGI
		\item Creazione di pagine di template per gli script Perl/CGI
		\item Codifica degli script Javascript
	\end{itemize}
	In generale tutti hanno inoltre contribuito alla correzione e all’individuazione dei \textit{bugs} e al suggerimento di nuove \textit{features} tramite le issues di Github.
\end{document}